%        File: resume.tex
%      Author: Leon Zaruvinsky
%     License: MIT License
%     Created: Fri Sep 05 06:00 PM 2014 E
% Last Change: Fri Sep 05 12:40 AM 2014 E
%
%
% There are two additionally defined types of subsection headers:
%
% \subsectiond{Arg 1}{Arg 2}
% Arg 1 is left aligned, Arg 2 is tabbed from Arg 1
%
% \subsectionp{Arg 1}{Arg 2}{Arg 3}
% Arg 1 is left aligned, Arg 2 is tabbed from Arg 1, and Arg 3 is right aligned
%
% See the sample PDF for a visual 
%
% % % % % % % % % % %
%
% \desc = description, aka paragraph
%
% % % % % % % % % % %
%
% This is my first LaTeX document, so I don't expect it to be pretty or perfect.
% Rather, I haven't been able to find a Resume template I liked so I made my own.
%
% If you find any errors or want to make improvements, feel free to submit a pull request.
%
%

\documentclass{resume}
\usepackage{tabularx}
\usepackage{hyphenat}
\usepackage{enumitem}

\begin{document}
\pagenumbering{gobble}
\name{John Harvey}

\info{\href{mailto:hello@email.com}{hello@email.com}}{800-555-1234}{\href{http://twitter.com/}{@Twitter}}{\href{http://github.com/}{GitHub.com/username}}

\vspace{-1.5em}
\section{Education}
\subsectionp{University Name, College of Engineering}{City, State}{Expected May 2017}
\desc{B.S. in Computer Science, Business minor, GPA 4.00}

\subsection{Selected Coursework}
\hspace{12pt} \begin{tabularx}{\linewidth}{r X}
    In Progress & Artificial Intelligence, Database Systems \\
    Completed & Operating Systems + Practicum, Analysis of Algorithms, Data Structures \& Functional Programming, Systems Programming, Digital Logic, Discrete Structures, Object Oriented Programming
\end{tabularx}

\vspace{-1em}

\section{Work Experience}

\subsectionp{Software Engineering Intern, Company}{City, State}{Summer 2015}
\desc{\begin{itemize}[leftmargin=*]
    \setlength\itemsep{-.1em}
    \item This is a standard format for a description.  No longer than one line per bullet.
    \item Lorem ipsum dolor sit amet, consectetur adipiscing elit. Sed quis tincidunt quam, suspendisse.
    \item Lorem ipsum dolor sit amet, consectetur adipiscing elit. Sed quis tincidunt quam, suspendisse.
\end{itemize}}

\subsectionp{Teaching Assistant}{}{Jan 2014 - Present}
\desc{This is a slightly different format for a job description.}
\hspace{3em} \begin{tabular}{l l}
    CS 4410: Operating Systems & 1 Semester \\
    CS 2110: Object Oriented Programming \& Data Structures \hspace{.5em} & 3 Semesters \\
    \end{tabular}


\subsectionp{Software Engineering Intern, Company}{City, State}{Summer 2014}
\desc{\begin{itemize}[leftmargin=*]
    \setlength\itemsep{-.1em}
    \item This is a standard format for a description.  No longer than one line per bullet.
    \item Lorem ipsum dolor sit amet, consectetur adipiscing elit. Sed quis tincidunt quam, suspendisse.
    \item Lorem ipsum dolor sit amet, consectetur adipiscing elit. Sed quis tincidunt quam, suspendisse.
\end{itemize}}

\section{Languages and Technologies}

\desc{
	\begin{description}
		\item[Most Experienced] Java, Python, C, HTML/CSS
		\item[Some Familiarity] OCaml, JavaScript (Node.js), Git, MySQL, MongoDB, Android, \LaTeX
	\end{description}
}

\section{Side Projects}

\subsectionp{Cool Personal Project}{}{Source: \href{http://github.com/}{github.com/}}
\desc{Lorem ipsum dolor sit amet, consectetur adipiscing elit. Sed quis tincidunt quam. Suspendisse neque felis, sagittis eget eros sed, feugiat aliquam nisl. In eu luctus nisl. In hac habitasse.}


\subsectionp{A Hackathon Project}{BigRed//Hacks}{Source: \href{http://github.com/}{github.com/}}
\desc{Lorem ipsum dolor sit amet, consectetur adipiscing elit. Sed quis tincidunt quam. Suspendisse neque felis, sagittis eget eros sed, feugiat aliquam nisl. In eu luctus nisl. In hac habitasse.}

\subsectionp{Some Other Thing}{Summer 2014}{Source: \href{http://github.com}{github.com/}}
\desc{Lorem ipsum dolor sit amet, consectetur adipiscing elit. Sed quis tincidunt quam. Suspendisse neque felis, sagittis eget eros sed, feugiat aliquam nisl. In eu luctus nisl. In hac habitasse.}

\section{Extracurricular Activities}

\subsectionp{A Committee or Club}{}{}
\desc{This is the place to put something that isn't explicitly coding but is still relevant to software! E.g., member of school's CS Undergrad group or Women in Computing.}

\end{document}

